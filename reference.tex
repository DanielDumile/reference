\documentclass[11pt]{article}
\usepackage[utf8]{inputenc}
\usepackage[spanish]{babel}
\decimalpoint
\usepackage{amsmath}
\usepackage{amsthm}
\usepackage{amssymb}
\usepackage{graphicx}
\usepackage[margin=0.8in,landscape]{geometry}
\usepackage{fancyhdr}
\usepackage[inline]{enumitem}
\usepackage{float}
\usepackage{cancel}
\usepackage{bigints}
\usepackage{listings}
\usepackage{xcolor}
\usepackage{listingsutf8}
\usepackage{algpseudocode}
\usepackage{algorithm}
\usepackage{apacite}
\usepackage{minted}
\usepackage{tcolorbox}
\usepackage{multicol}
\usepackage{hyperref}
\hypersetup{
	colorlinks,
	citecolor=black,
	filecolor=black,
	linkcolor=black,
	urlcolor=black
}
\twocolumn
\usepackage{tipa}
\pagestyle{fancy}
\newcommand{\xvdash}[1]{%
	\vdash^{\mkern-10mu\scriptscriptstyle\rule[-.9ex]{0pt}{0pt}#1}%
}
\setlength{\headheight}{15pt} 
\lhead{Reference}
\rhead{\thepage}
\lfoot{ESCOM-IPN}
\renewcommand{\footrulewidth}{0.5pt}
\setlength{\parskip}{0.5em}
\newcommand{\ve}[1]{\overrightarrow{#1}}
\newcommand{\abs}[1]{\left\lvert #1 \right\lvert}
\newcommand{\blank}{\text{\textcrb}}
\title{Reference}

\lstdefinestyle{customc}{
	belowcaptionskip=1\baselineskip,
	breaklines=true,
	frame=L,
	xleftmargin=\parindent,
	language=C++,
	showstringspaces=false,
	basicstyle=\ttfamily,
	keywordstyle=\bfseries\color{green!40!black},
	commentstyle=\itshape\color{purple!40!black},
	identifierstyle=\color{blue},
	numbers=left,
	stringstyle=\color{orange},
}

\begin{document}
	\tableofcontents
	
	\clearpage
	\section{Teoría de números}
		\subsection{Funciones básicas}
			\subsubsection{Función piso y techo}
			\inputminted[tabsize=2,breaklines,firstline=5,lastline=21,fontsize=\small]{c++}{numberTheory.cpp}
			
			\subsubsection{Exponenciación y multiplicación binaria}
			\inputminted[tabsize=2,breaklines,firstline=23,lastline=46,fontsize=\small]{c++}{numberTheory.cpp}
			
			\subsubsection{Mínimo común múltiplo y máximo común divisor}
			\inputminted[tabsize=2,breaklines,firstline=48,lastline=68,fontsize=\small]{c++}{numberTheory.cpp}
			
			\subsubsection{Euclides extendido e inverso modular}
			\inputminted[tabsize=2,breaklines,firstline=70,lastline=91,fontsize=\small]{c++}{numberTheory.cpp}
			
			\subsubsection{Todos los inversos módulo $p$}
			\inputminted[tabsize=2,breaklines,firstline=711,lastline=718,fontsize=\small]{c++}{numberTheory.cpp}
			
			\subsubsection{Exponenciación binaria modular}
			\inputminted[tabsize=2,breaklines,firstline=93,lastline=106,fontsize=\small]{c++}{numberTheory.cpp}
			
			\subsubsection{Teorema chino del residuo}
			\inputminted[tabsize=2,breaklines,firstline=108,lastline=117,fontsize=\small]{c++}{numberTheory.cpp}
			
			\subsubsection{Teorema chino del residuo generalizado}
			\inputminted[tabsize=2,breaklines,firstline=950,lastline=992,fontsize=\small]{c++}{numberTheory.cpp}
			
			\subsubsection{Coeficiente binomial}
			\inputminted[tabsize=2,breaklines,firstline=433,lastline=440,fontsize=\small]{c++}{numberTheory.cpp}
			
			\subsubsection{Fibonacci}
			\inputminted[tabsize=2,breaklines,firstline=720,lastline=741,fontsize=\small]{c++}{numberTheory.cpp}
		
		\subsection{Cribas}
			\subsubsection{Criba de divisores}
			\inputminted[tabsize=2,breaklines,firstline=119,lastline=130,fontsize=\small]{c++}{numberTheory.cpp}
			
			\subsubsection{Criba de primos}
			\inputminted[tabsize=2,breaklines,firstline=132,lastline=148,fontsize=\small]{c++}{numberTheory.cpp}
			
			\subsubsection{Criba de factor primo más pequeño}
			\inputminted[tabsize=2,breaklines,firstline=150,lastline=160,fontsize=\small]{c++}{numberTheory.cpp}
			
			\subsubsection{Criba de factor primo más grande}
			\inputminted[tabsize=2,breaklines,firstline=939,lastline=948,fontsize=\small]{c++}{numberTheory.cpp}
			
			\subsubsection{Criba de factores primos}
			\inputminted[tabsize=2,breaklines,firstline=162,lastline=170,fontsize=\small]{c++}{numberTheory.cpp}
			
			\subsubsection{Criba de la función $\varphi$ de Euler}
			\inputminted[tabsize=2,breaklines,firstline=172,lastline=180,fontsize=\small]{c++}{numberTheory.cpp}
			
			\subsubsection{Criba de la función $\mu$}
			\inputminted[tabsize=2,breaklines,firstline=801,lastline=809,fontsize=\small]{c++}{numberTheory.cpp}
			
			\subsubsection{Triángulo de Pascal}
			\inputminted[tabsize=2,breaklines,firstline=182,lastline=192,fontsize=\small]{c++}{numberTheory.cpp}
			
			\subsubsection{Segmented sieve}
			\inputminted[tabsize=2,breaklines,firstline=860,lastline=892,fontsize=\small]{c++}{numberTheory.cpp}
			
			\subsubsection{Criba de primos lineal}
			\inputminted[tabsize=2,breaklines,firstline=811,lastline=825,fontsize=\small]{c++}{numberTheory.cpp}
			
			\subsubsection{Criba lineal para funciones multiplicativas}
			\inputminted[tabsize=2,breaklines,firstline=827,lastline=858,fontsize=\small]{c++}{numberTheory.cpp}
			
		
		\subsection{Factorización}
			\subsubsection{Factorización de un número}
			\inputminted[tabsize=2,breaklines,firstline=194,lastline=207,fontsize=\small]{c++}{numberTheory.cpp}
			
			\subsubsection{Potencia de un primo que divide a un factorial}
			\inputminted[tabsize=2,breaklines,firstline=418,lastline=422,fontsize=\small]{c++}{numberTheory.cpp}
			
			\subsubsection{Factorización de un factorial}
			\inputminted[tabsize=2,breaklines,firstline=424,lastline=431,fontsize=\small]{c++}{numberTheory.cpp}
			
			\subsubsection{Factorización usando Pollard-Rho}
			\inputminted[tabsize=2,breaklines,firstline=657,lastline=705,fontsize=\small]{c++}{numberTheory.cpp}
		
		\subsection{Funciones aritméticas famosas}
			\subsubsection{Función $\sigma$}
			\inputminted[tabsize=2,breaklines,firstline=209,lastline=226,fontsize=\small]{c++}{numberTheory.cpp}
			
			\subsubsection{Función $\Omega$}
			\inputminted[tabsize=2,breaklines,firstline=228,lastline=235,fontsize=\small]{c++}{numberTheory.cpp}
			
			\subsubsection{Función $\omega$}
			\inputminted[tabsize=2,breaklines,firstline=237,lastline=244,fontsize=\small]{c++}{numberTheory.cpp}
			
			\subsubsection{Función $\varphi$ de Euler}
			\inputminted[tabsize=2,breaklines,firstline=251,lastline=258,fontsize=\small]{c++}{numberTheory.cpp}
			
			\subsubsection{Función $\mu$}
			\inputminted[tabsize=2,breaklines,firstline=276,lastline=287,fontsize=\small]{c++}{numberTheory.cpp}
			
		\subsection{Orden multiplicativo, raíces primitivas y raíces de la unidad}
			\subsubsection{Función $\lambda$ de Carmichael}
			\inputminted[tabsize=2,breaklines,firstline=260,lastline=274,fontsize=\small]{c++}{numberTheory.cpp}
			
			\subsubsection{Orden multiplicativo módulo $m$}
			\inputminted[tabsize=2,breaklines,firstline=289,lastline=305,fontsize=\small]{c++}{numberTheory.cpp}
			
			\subsubsection{Número de raíces primitivas (generadores) módulo $m$}
			\inputminted[tabsize=2,breaklines,firstline=307,lastline=313,fontsize=\small]{c++}{numberTheory.cpp}
			
			\subsubsection{Test individual de raíz primitiva módulo $m$}
			\inputminted[tabsize=2,breaklines,firstline=315,lastline=325,fontsize=\small]{c++}{numberTheory.cpp}
			
			\subsubsection{Test individual de raíz $k$-ésima de la unidad módulo $m$}
			\inputminted[tabsize=2,breaklines,firstline=327,lastline=336,fontsize=\small]{c++}{numberTheory.cpp}
			
			\subsubsection{Encontrar la primera raíz primitiva módulo $m$}
			\inputminted[tabsize=2,breaklines,firstline=338,lastline=355,fontsize=\small]{c++}{numberTheory.cpp}
			
			\subsubsection{Encontrar la primera raíz $k$-ésima de la unidad módulo $m$}
			\inputminted[tabsize=2,breaklines,firstline=357,lastline=373,fontsize=\small]{c++}{numberTheory.cpp}
			
			\subsubsection{Logaritmo discreto}
			\inputminted[tabsize=2,breaklines,firstline=375,lastline=398,fontsize=\small]{c++}{numberTheory.cpp}
			
			\subsubsection{Raíz $k$-ésima discreta}
			\inputminted[tabsize=2,breaklines,firstline=400,lastline=416,fontsize=\small]{c++}{numberTheory.cpp}
			
			\subsubsection{Algoritmo de Tonelli-Shanks para raíces cuadradas módulo $p$}
			\inputminted[tabsize=2,breaklines,firstline=908,lastline=937,fontsize=\small]{c++}{numberTheory.cpp}
			
		\subsection{Particiones}
			\subsubsection{Función $P$ (particiones de un entero positivo)}
			\inputminted[tabsize=2,breaklines,firstline=519,lastline=547,fontsize=\small]{c++}{numberTheory.cpp}
			
			\subsubsection{Función $Q$ (particiones de un entero positivo en distintos sumandos)}
			\inputminted[tabsize=2,breaklines,firstline=549,lastline=596,fontsize=\small]{c++}{numberTheory.cpp}
			
			\subsubsection{Número de factorizaciones ordenadas}
			\inputminted[tabsize=2,breaklines,firstline=743,lastline=771,fontsize=\small]{c++}{numberTheory.cpp}
			
			\subsubsection{Número de factorizaciones no ordenadas}
			\inputminted[tabsize=2,breaklines,firstline=773,lastline=799,fontsize=\small]{c++}{numberTheory.cpp}
			
		\subsection{Otros}
			\subsubsection{Cambio de base}
			\inputminted[tabsize=2,breaklines,firstline=442,lastline=462,fontsize=\small]{c++}{numberTheory.cpp}
			
			\subsubsection{Fracciones continuas}
			\inputminted[tabsize=2,breaklines,firstline=598,lastline=640,fontsize=\small]{c++}{numberTheory.cpp}
			
			\subsubsection{Ecuación de Pell}
			\inputminted[tabsize=2,breaklines,firstline=642,lastline=655,fontsize=\small]{c++}{numberTheory.cpp}
			
			\subsubsection{Números de Bell}
			\inputminted[tabsize=2,breaklines,firstline=894,lastline=906,fontsize=\small]{c++}{numberTheory.cpp}
			
			\subsubsection{Prime counting function in sublinear time}
			\inputminted[tabsize=2,breaklines,firstline=27,lastline=77,fontsize=\small]{c++}{pi.cpp}
			
	\newpage
	\section{Números racionales}
		\subsection{Estructura \texttt{fraccion}}
		\inputminted[tabsize=2,breaklines,firstline=7,lastline=123,fontsize=\small]{c++}{fraccion.cpp}
		
	\newpage
	\section{Álgebra lineal}
		\subsection{Estructura \texttt{matrix}}
		\inputminted[tabsize=2,breaklines,firstline=7,lastline=130,fontsize=\small]{c++}{matrix.cpp}
		
		\subsection{Transpuesta y traza}
		\inputminted[tabsize=2,breaklines,firstline=132,lastline=145,fontsize=\small]{c++}{matrix.cpp}
		
		\subsection{Gauss Jordan}
		\inputminted[tabsize=2,breaklines,firstline=147,lastline=190,fontsize=\small]{c++}{matrix.cpp}
		
		\subsection{Matriz escalonada por filas y reducida por filas}
		\inputminted[tabsize=2,breaklines,firstline=192,lastline=202,fontsize=\small]{c++}{matrix.cpp}
		
		\subsection{Matriz inversa}
		\inputminted[tabsize=2,breaklines,firstline=204,lastline=225,fontsize=\small]{c++}{matrix.cpp}
		
		\subsection{Determinante}
		\inputminted[tabsize=2,breaklines,firstline=227,lastline=240,fontsize=\small]{c++}{matrix.cpp}
		
		\subsection{Matriz de cofactores y adjunta}
		\inputminted[tabsize=2,breaklines,firstline=242,lastline=267,fontsize=\small]{c++}{matrix.cpp}
		
		\subsection{Factorización $PA=LU$}
		\inputminted[tabsize=2,breaklines,firstline=269,lastline=285,fontsize=\small]{c++}{matrix.cpp}
		
		\subsection{Polinomio característico}
		\inputminted[tabsize=2,breaklines,firstline=287,lastline=297,fontsize=\small]{c++}{matrix.cpp}
		
		\subsection{Gram-Schmidt}
		\inputminted[tabsize=2,breaklines,firstline=299,lastline=315,fontsize=\small]{c++}{matrix.cpp}
		
		\subsection{Recurrencias lineales}
		\inputminted[tabsize=2,breaklines,firstline=7,lastline=38,fontsize=\small]{c++}{recurrence.cpp}
		
		\subsection{Simplex}
		\inputminted[tabsize=2,breaklines,fontsize=\small]{c++}{simplex.cpp}
		
		
	\newpage
	\section{FFT}
		\subsection{Funciones previas}
		\inputminted[tabsize=2,breaklines,firstline=3,lastline=11,fontsize=\small]{c++}{fft.cpp}
		
		\subsection{FFT con raíces de la unidad complejas}
		\inputminted[tabsize=2,breaklines,firstline=13,lastline=44,fontsize=\small]{c++}{fft.cpp}
		
		\subsection{FFT con raíces de la unidad discretas (NTT)}
		\inputminted[tabsize=2,breaklines,firstline=46,lastline=90,fontsize=\small]{c++}{fft.cpp}
			\subsubsection{Otros valores para escoger la raíz y el módulo}
				\begin{table}[H]
					\centering
					\begin{tabular}{|p{2cm}|p{1.7cm}|p{2cm}|p{4.5cm}|}
						\hline
						Raíz $n$-ésima de la unidad ($\omega$) & $\omega^{-1}$ & Tamaño máximo del arreglo ($n$) & Módulo $p$ \\ \hline
						15 & 30584 & $2^{14}$ & $4 \times 2^{14} + 1 = 65537$ \\ \hline
						9 & 7282 & $2^{15}$ & $2 \times 2^{15} + 1 = 65537$ \\ \hline
						3 & 21846 & $2^{16}$ & $1 \times 2^{16} + 1 = 65537$ \\ \hline
						8 & 688129 & $2^{17}$ & $6 \times 2^{17} + 1 = 786433$ \\ \hline
						5 & 471860 & $2^{18}$ & $3 \times 2^{18} + 1 = 786433$ \\ \hline
						12 & 3364182 & $2^{19}$ & $11 \times 2^{19} + 1 = 5767169$ \\ \hline
						\textbf{5} & \textbf{4404020} & $\mathbf{2^{20}}$ & $7 \times 2^{20} + 1 = \textbf{7340033}$ \\ \hline
						38 & 21247462 & $2^{21}$ & $11 \times 2^{21} + 1 = 23068673$ \\ \hline
						21 & 49932191 & $2^{22}$ & $25 \times 2^{22} + 1 = 104857601$ \\ \hline
						4 & 125829121 & $2^{23}$ & $20 \times 2^{23} + 1 = 167772161$ \\ \hline
						\textbf{31} & \textbf{128805723} & $\mathbf{2^{23}}$ & $119 \times 2^{23} + 1 = \textbf{998244353}$ \\ \hline
						2 & 83886081 & $2^{24}$ & $10 \times 2^{24} + 1 = 167772161$ \\ \hline
						17 & 29606852 & $2^{25}$ & $5 \times 2^{25} + 1 = 167772161$ \\ \hline
						30 & 15658735 & $2^{26}$ & $7 \times 2^{26} + 1 = 469762049$ \\ \hline
						137 & 749463956 & $2^{27}$ & $15 \times 2^{27} + 1 = 2013265921$ \\ \hline
					\end{tabular}
				\end{table}
		
		\subsection{Aplicaciones}
			\subsubsection{Multiplicación de polinomios (convolución lineal)}
			\inputminted[tabsize=2,breaklines,firstline=92,lastline=118,fontsize=\small]{c++}{fft.cpp}
			
			\subsubsection{Multiplicación de números enteros grandes}
			\inputminted[tabsize=2,breaklines,firstline=120,lastline=156,fontsize=\small]{c++}{fft.cpp}
			
			\subsubsection{Inverso de un polinomio}
			\inputminted[tabsize=2,breaklines,firstline=158,lastline=184,fontsize=\small]{c++}{fft.cpp}
			
			\subsubsection{Raíz cuadrada de un polinomio}
			\inputminted[tabsize=2,breaklines,firstline=186,lastline=208,fontsize=\small]{c++}{fft.cpp}
			
		\subsection{FFT con tamaño de vector arbitrario (algoritmo de Bluestein)}
		\inputminted[tabsize=2,breaklines,firstline=210,lastline=224,fontsize=\small]{c++}{fft.cpp}
			
	\newpage
	\section{Geometría}
		\subsection{Estructura \texttt{point}}
		\inputminted[tabsize=2,breaklines,firstline=4,lastline=97,fontsize=\small]{c++}{geometry.cpp}
		
		\subsection{Líneas y segmentos}
			\subsubsection{Verificar si un punto pertenece a una línea o segmento}
			\inputminted[tabsize=2,breaklines,firstline=102,lastline=111,fontsize=\small]{c++}{geometry.cpp}
			
			\subsubsection{Intersección de líneas}
			\inputminted[tabsize=2,breaklines,firstline=113,lastline=133,fontsize=\small]{c++}{geometry.cpp}
			
			\subsubsection{Intersección línea-segmento}
			\inputminted[tabsize=2,breaklines,firstline=135,lastline=148,fontsize=\small]{c++}{geometry.cpp}
			
			\subsubsection{Intersección de segmentos}
			\inputminted[tabsize=2,breaklines,firstline=150,lastline=167,fontsize=\small]{c++}{geometry.cpp}
			
			\subsubsection{Distancia punto-recta}
			\inputminted[tabsize=2,breaklines,firstline=169,lastline=172,fontsize=\small]{c++}{geometry.cpp}
			
		\subsection{Círculos}
			\subsubsection{Distancia punto-círculo}
			\inputminted[tabsize=2,breaklines,firstline=391,lastline=394,fontsize=\small]{c++}{geometry.cpp}
			
			\subsubsection{Proyección punto exterior a círculo}
			\inputminted[tabsize=2,breaklines,firstline=396,lastline=399,fontsize=\small]{c++}{geometry.cpp}
			
			\subsubsection{Puntos de tangencia de punto exterior}
			\inputminted[tabsize=2,breaklines,firstline=401,lastline=406,fontsize=\small]{c++}{geometry.cpp}
			
			\subsubsection{Intersección línea-círculo}
			\inputminted[tabsize=2,breaklines,firstline=408,lastline=422,fontsize=\small]{c++}{geometry.cpp}
			
			\subsubsection{Centro y radio a través de tres puntos}
			\inputminted[tabsize=2,breaklines,firstline=424,lastline=429,fontsize=\small]{c++}{geometry.cpp}
			
			\subsubsection{Intersección de círculos}
			\inputminted[tabsize=2,breaklines,firstline=431,lastline=448,fontsize=\small]{c++}{geometry.cpp}
			
			\subsubsection{Contención de círculos}
			\inputminted[tabsize=2,breaklines,firstline=450,lastline=469,fontsize=\small]{c++}{geometry.cpp}
			
			\subsubsection{Tangentes}
			\inputminted[tabsize=2,breaklines,firstline=471,lastline=501,fontsize=\small]{c++}{geometry.cpp}
			
			\subsubsection{Smallest enclosing circle}
			\inputminted[tabsize=2,breaklines,firstline=756,lastline=787,fontsize=\small]{c++}{geometry.cpp}
		
		\subsection{Polígonos}
			\subsubsection{Perímetro y área de un polígono}
			\inputminted[tabsize=2,breaklines,firstline=174,lastline=190,fontsize=\small]{c++}{geometry.cpp}
			
			\subsubsection{Envolvente convexa (convex hull) de un polígono}
			\inputminted[tabsize=2,breaklines,firstline=192,lastline=211,fontsize=\small]{c++}{geometry.cpp}
			
			\subsubsection{Verificar si un punto pertenece al perímetro de un polígono}
			\inputminted[tabsize=2,breaklines,firstline=213,lastline=221,fontsize=\small]{c++}{geometry.cpp}
			
			\subsubsection{Verificar si un punto pertenece a un polígono}
			\inputminted[tabsize=2,breaklines,firstline=223,lastline=234,fontsize=\small]{c++}{geometry.cpp}
			
			\subsubsection{Verificar si un punto pertenece a un polígono convexo $O(\log n)$}
			\inputminted[tabsize=2,breaklines,firstline=559,lastline=586,fontsize=\small]{c++}{geometry.cpp}
			
			\subsubsection{Cortar un polígono con una recta}
			\inputminted[tabsize=2,breaklines,firstline=526,lastline=557,fontsize=\small]{c++}{geometry.cpp}
			
			\subsubsection{Centroide de un polígono}
			\inputminted[tabsize=2,breaklines,firstline=264,lastline=274,fontsize=\small]{c++}{geometry.cpp}
			
			\subsubsection{Pares de puntos antipodales}
			\inputminted[tabsize=2,breaklines,firstline=341,lastline=352,fontsize=\small]{c++}{geometry.cpp}
			
			\subsubsection{Diámetro y ancho}
			\inputminted[tabsize=2,breaklines,firstline=354,lastline=368,fontsize=\small]{c++}{geometry.cpp}
			
			\subsubsection{Smallest enclosing rectangle}
			\inputminted[tabsize=2,breaklines,firstline=370,lastline=389,fontsize=\small]{c++}{geometry.cpp}
		
		\subsection{Par de puntos más cercanos}
		\inputminted[tabsize=2,breaklines,firstline=236,lastline=262,fontsize=\small]{c++}{geometry.cpp}
		
		\subsection{Vantage Point Tree (puntos más cercanos a cada punto)}
		\inputminted[tabsize=2,breaklines,firstline=276,lastline=339,fontsize=\small]{c++}{geometry.cpp}
		
		\subsection{Suma Minkowski}
		\inputminted[tabsize=2,breaklines,firstline=503,lastline=524,fontsize=\small]{c++}{geometry.cpp}
		
		\subsection{Triangulación de Delaunay}
		\inputminted[tabsize=2,breaklines,firstline=588,lastline=754,fontsize=\small]{c++}{geometry.cpp}
		
	\newpage
	\section{Grafos}
		\subsection{Disjoint Set}
		\inputminted[tabsize=2,breaklines,firstline=8,lastline=37,fontsize=\small]{c++}{graph.cpp}
		
		\subsection{Definiciones}
		\inputminted[tabsize=2,breaklines,firstline=39,lastline=100,fontsize=\small]{c++}{graph.cpp}
		
		\subsection{DFS genérica}
		\inputminted[tabsize=2,breaklines,firstline=411,lastline=429,fontsize=\small]{c++}{graph.cpp}
		
		\subsection{Dijkstra}
		\inputminted[tabsize=2,breaklines,firstline=102,lastline=125,fontsize=\small]{c++}{graph.cpp}
		
		\subsection{Bellman Ford}
		\inputminted[tabsize=2,breaklines,firstline=127,lastline=161,fontsize=\small]{c++}{graph.cpp}
		
		\subsection{Floyd}
		\inputminted[tabsize=2,breaklines,firstline=167,lastline=175,fontsize=\small]{c++}{graph.cpp}
		
		\subsection{Cerradura transitiva $O(V^3)$}
		\inputminted[tabsize=2,breaklines,firstline=177,lastline=184,fontsize=\small]{c++}{graph.cpp}
		
		\subsection{Cerradura transitiva $O(V^2)$}
		\inputminted[tabsize=2,breaklines,firstline=186,lastline=200,fontsize=\small]{c++}{graph.cpp}
		
		\subsection{Verificar si el grafo es bipartito}
		\inputminted[tabsize=2,breaklines,firstline=202,lastline=224,fontsize=\small]{c++}{graph.cpp}
		
		\subsection{Orden topológico}
		\inputminted[tabsize=2,breaklines,firstline=226,lastline=252,fontsize=\small]{c++}{graph.cpp}

		\subsection{Detectar ciclos}
		\inputminted[tabsize=2,breaklines,firstline=254,lastline=274,fontsize=\small]{c++}{graph.cpp}
		
		\subsection{Puentes y puntos de articulación}
		\inputminted[tabsize=2,breaklines,firstline=276,lastline=304,fontsize=\small]{c++}{graph.cpp}
		
		\subsection{Componentes fuertemente conexas}
		\inputminted[tabsize=2,breaklines,firstline=306,lastline=335,fontsize=\small]{c++}{graph.cpp}
		
		\subsection{Árbol mínimo de expansión (Kruskal)}
		\inputminted[tabsize=2,breaklines,firstline=337,lastline=353,fontsize=\small]{c++}{graph.cpp}
		
		\subsection{Máximo emparejamiento bipartito}
		\inputminted[tabsize=2,breaklines,firstline=355,lastline=409,fontsize=\small]{c++}{graph.cpp}
		
		\subsection{Circuito euleriano}
		
		
	\newpage
	\section{Árboles}		
		\subsection{Estructura \texttt{tree}}
		\inputminted[tabsize=2,breaklines,firstline=432,lastline=470,fontsize=\small]{c++}{graph.cpp}
		
		\subsection{$k$-ésimo ancestro}
		\inputminted[tabsize=2,breaklines,firstline=472,lastline=484,fontsize=\small]{c++}{graph.cpp}
		
		\subsection{LCA}
		\inputminted[tabsize=2,breaklines,firstline=486,lastline=505,fontsize=\small]{c++}{graph.cpp}
		
		\subsection{Distancia entre dos nodos}
		\inputminted[tabsize=2,breaklines,firstline=507,lastline=530,fontsize=\small]{c++}{graph.cpp}
		
		\subsection{HLD}
		
		
		\subsection{Link Cut}
		
		
	\newpage
	\section{Flujos}
		\subsection{Estructura \texttt{flowEdge}}
		\inputminted[tabsize=2,breaklines,firstline=4,lastline=17,fontsize=\small]{c++}{flow.cpp}
		
		\subsection{Estructura \texttt{flowGraph}}
		\inputminted[tabsize=2,breaklines,firstline=19,lastline=38,fontsize=\small]{c++}{flow.cpp}
		
		\subsection{Algoritmo de Edmonds-Karp $O(VE^2)$}
		\inputminted[tabsize=2,breaklines,firstline=82,lastline=108,fontsize=\small]{c++}{flow.cpp}
		
		\subsection{Algoritmo de Dinic $O(V^2E)$}
		\inputminted[tabsize=2,breaklines,firstline=40,lastline=80,fontsize=\small]{c++}{flow.cpp}
		
		\subsection{Flujo máximo de costo mínimo}
		\inputminted[tabsize=2,breaklines,firstline=110,lastline=145,fontsize=\small]{c++}{flow.cpp}
		
	\newpage
	\section{Estructuras de datos}
		\subsection{Segment Tree}
			\subsubsection{Minimalistic: Point updates, range queries}
			\inputminted[tabsize=2,breaklines,firstline=4,lastline=47,fontsize=\small]{c++}{queries.cpp}
			
			\subsubsection{Dynamic: Range updates and range queries}
			\inputminted[tabsize=2,breaklines,firstline=49,lastline=100,fontsize=\small]{c++}{queries.cpp}
			
			\subsubsection{Static: Range updates and range queries}
			\inputminted[tabsize=2,breaklines,firstline=102,lastline=171,fontsize=\small]{c++}{queries.cpp}
			
			\subsubsection{Persistent: Point updates, range queries}
			\inputminted[tabsize=2,breaklines,firstline=173,lastline=203,fontsize=\small]{c++}{queries.cpp}
		
		\subsection{Fenwick Tree}
		\inputminted[tabsize=2,breaklines,firstline=205,lastline=242,fontsize=\small]{c++}{queries.cpp}
		
		\subsection{SQRT Decomposition}
		\inputminted[tabsize=2,breaklines,firstline=244,lastline=322,fontsize=\small]{c++}{queries.cpp}
		
		\subsection{AVL Tree}
		\inputminted[tabsize=2,breaklines,firstline=324,lastline=529,fontsize=\small]{c++}{queries.cpp}
		
		\subsection{Treap}
		\inputminted[tabsize=2,breaklines,firstline=531,lastline=796,fontsize=\small]{c++}{queries.cpp}
		
		\subsection{Sparse table}
			\subsubsection{Normal}
			\inputminted[tabsize=2,breaklines,firstline=798,lastline=833,fontsize=\small]{c++}{queries.cpp}
		
			\subsection{Disjoint}
			\inputminted[tabsize=2,breaklines,firstline=835,lastline=868,fontsize=\small]{c++}{queries.cpp}
			
		\subsection{Wavelet Tree}
		\inputminted[tabsize=2,breaklines,firstline=870,lastline=933,fontsize=\small]{c++}{queries.cpp}
		
		\subsection{Ordered Set C++}
		\inputminted[tabsize=2,breaklines,firstline=935,lastline=969,fontsize=\small]{c++}{queries.cpp}
		
		\subsection{Splay Tree}
		
		
		\subsection{Red Black Tree}
		
		
	\newpage
	\section{Cadenas}
		\subsection{Trie}
		\inputminted[tabsize=2,breaklines,firstline=144,lastline=196,fontsize=\small]{c++}{strings.cpp}
		
		\subsection{KMP}
		\inputminted[tabsize=2,breaklines,firstline=4,lastline=39,fontsize=\small]{c++}{strings.cpp}
		
		\subsection{Aho-Corasick}
		\inputminted[tabsize=2,breaklines,firstline=41,lastline=142,fontsize=\small]{c++}{strings.cpp}
		
		\subsection{Rabin-Karp}
		
		
		\subsection{Suffix Array}
		
		
		\subsection{Función Z}
		
	
	\newpage
	\section{Varios}
		\subsection{Lectura y escritura de \texttt{\_\_int128}}
		\inputminted[tabsize=2,breaklines,firstline=46,lastline=83,fontsize=\small]{c++}{misc.cpp}
		
		\subsection{Longest Common Subsequence (LCS)}
		\inputminted[tabsize=2,breaklines,firstline=21,lastline=33,fontsize=\small]{c++}{misc.cpp}
		
		\subsection{Longest Increasing Subsequence (LIS)}
		\inputminted[tabsize=2,breaklines,firstline=5,lastline=19,fontsize=\small]{c++}{misc.cpp}
		
		\subsection{Levenshtein Distance}
		\inputminted[tabsize=2,breaklines,firstline=145,lastline=156,fontsize=\small]{c++}{misc.cpp}
		
		\subsection{Día de la semana}
		\inputminted[tabsize=2,breaklines,firstline=35,lastline=44,fontsize=\small]{c++}{misc.cpp}
		
		\subsection{2SAT}
		\inputminted[tabsize=2,breaklines,firstline=85,lastline=128,fontsize=\small]{c++}{misc.cpp}
		
		\subsection{Código Gray}
		\inputminted[tabsize=2,breaklines,firstline=130,lastline=143,fontsize=\small]{c++}{misc.cpp}
		
		\subsection{Contar número de unos en binario en un rango}
		\inputminted[tabsize=2,breaklines,firstline=158,lastline=165,fontsize=\small]{c++}{misc.cpp}

\end{document}